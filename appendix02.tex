\section{Code usage}

Here we document the various command line options for the code and its use in a variety of situations. All the command
line options for \lppen, as shown with the \verb|--help| command, are given below.
\onecolumngrid
\begin{footnotesize}
\begin{verbatim}
Usage: lalapps_pulsar_parameter_estimation_nested [options]

 --help              display this message
 --verbose           display all error messages
 --detectors         all IFOs with data to be analysed e.g. H1,H2 (delimited by commas) (if generating fake data these
                     should not be set)
 --par-file          pulsar parameter (.par) file (full path)
 --cor-file          pulsar TEMPO-fit parameter correlation matrix
 --input-files       full paths and file names for the data for each detector and model harmonic in the list (must be
                     in the same order) delimited by commas. These files can be gzipped. If not set you can generate
                     fake data (see --fake-data below)
 --sample-interval   (REAL8) the time interval bewteen samples (default to 60 s)
 --outfile           name of output data file (a HDF5 formated file with the extension '.hdf' or '.h5' [required])
 --output-chunks     Output lists of stationary chunks into which the data has been split
 --outXML            name of output XML file [not required]
 --chunk-min         (INT4) minimum stationary length of data to be used in the likelihood e.g. 5 mins
 --chunk-max         (INT4) maximum stationary length of data to be used in the likelihood e.g. 30 mins
 --time-bins         (INT4) no. of time bins in the time-psi lookup table
 --prior-file        file containing the parameters to search over and their upper and lower ranges
 --ephem-earth       Earth ephemeris file
 --ephem-sun         Sun ephemeris file
 --ephem-timecorr    Einstein delay time correction ephemeris file
 --harmonics         (CHAR) the signal model frequency harmonics that you want to use (delimited by commas). Currently
                     this can be either the 'triaxial' model for which you use 2 (the default value) or 1,2 for a
                     model with emission at both the rotation frequency and twice the rotation frequency.
 --biaxial           Set this if the waveform model parameters spcify a biaxial star
 --gaussian-like     Set this if a Gaussian likelihood is to be used. If the input file contains a column specifying
                     the noise standard deviation of the data then that will be used in the Gaussian likelihood
                     function, otherwise the noise variance will be calculated from the data.
 --nonGR             Set to allow non-GR polarisation modes and/or a variable speed of gravitational waves
 --randomise         Set this, with an INT seed, to randomise the data (through permutations of the time stamps) for
                     use in Monte-Carlo studies. NOTE: this will not work if using the code to create injections

 Nested sampling parameters:
 --Nlive             (INT4) no. of live points for nested sampling
 --Nmcmc             (INT4) no. of for MCMC used to find new live points (if not specified an  adaptive number of
                     points is used)
 --Nmcmcinitial      (INT4) no. of MCMC points to use in the initial resampling of the prior (default is to use
                     MAXMCMC)
 --Nruns             (INT4) no. of parallel runs
 --tolerance         (REAL8) tolerance of nested sampling integrator
 --randomseed        seed for random number generator

 MCMC proposal parameters:
 --diffev            (UINT4) relative weight of using differential evolution of the live points as the proposal
                     (DEFAULT = 0, e.g. 0%)
 --freqBinJump       (UINT4) relative weight of using jumps to adjacent frequency bins as a proposal (DEFAULT = 0,
                     e.g. this is not required unless searching over frequency)
 --ensembleStretch   (UINT4) relative weight of the ensemble stretch proposal (DEFAULT = 0, e.g. 0%) [NOTE: this
                     proposal greatly increases the autocorrelation lengths, so in general should be avoided]
 --ensembleWalk      (UINT4) relative weight of the ensemble walk proposal (DEFAULT = 3, e.g. 75%)
 --uniformprop       (UINT4) relative weights of uniform proposal (DEFAULT = 1, e.g. 25%)

 Reduced order quadrature (ROQ) parameters:
 --roq               Set this to use reduced order quadrature to compute the likelihood
 --ntraining         (UINT4) The number of training models used to generate an orthonormal basis of waveform models
 --roq-tolerance     (REAL8) The tolerance used during the basis generation (DEFAULT = 1e-11)
 --enrich-max        (UINT4) The number of times to try and "enrich" the basis set using new training data. The
                     enrichment process stop before this value is reached if three consecutive enrichment steps
                     produce no new bases (DEFAULT = 100)
 --roq-uniform       Set this flag to cause training model parameters for parameters with Gaussian prior
                     distributions to be drawn from a uniform distribution spanning mu +/- 5 sigma. Otherwise, by
                     default parameters are drawn from their given prior distributions
 --output-weights    (CHAR) If this is set then the weights will be output to the (binary) file that is named and the
                     programme will exit. These could be read in later instead of being regenerated. This allows the
                     ROQ to be generated on a machine with a large amount of RAM, whilst the full parameter
                     estimation can run on a machine with less RAM.
 --input-weights     (CHAR) A binary file containing all the weights in a defined format. If this is present then the
                     RQO will not be recalculated

 Signal injection parameters:
 --inject-file       a pulsar parameter (par) file containing the parameters of a signal to be injected. If this is
                     given a signal will be injected
 --inject-output     a filename to which the injected signal will be output if specified
 --fake-data         a list of IFO's for which fake data will be generated e.g. H1,L1 (delimited by commas). Unless
                     the --fake-psd flag is set the power spectral density for the data will be generated from the
                     noise models in LALNoiseModels. For Advanced detectors (e.g Advanced LIGO) prefix the name with
                     an A (e.g. AH1 for an Advanced LIGO detector at Hanford). The noise will be white across the data
                     bandwidth
 --fake-psd          if you want to generate fake data with specific power spectral densities for each detector giving
                     in --fake-data then they should be specified here delimited by commas (e.g. for --fake-data H1,L1
                     then you could use --fake-psd 1e-48,1.5e-48) where values are single-sided PSDs in Hz^-1
 --fake-starts       the start times (in GPS seconds) of the fake data for each detector separated by commas (e.g.
                     910000000,910021000). If not specified these will all default to 900000000
 --fake-lengths      the length of each fake data set (in seconds) for each detector separated by commas. If not
                     specified these will all default to 86400 (i.e. 1 day)
 --fake-dt           the data sample rate (in seconds) for the fake data for each detector. If not specified this will
                     default to 60s
 --scale-snr         give a (multi-detector) SNR value to which you want to scale the injection. This is 1 by default

 Legacy code flags:
 --oldChunks        Set if using fixed chunk sizes for dividing the data as in the old code, rather than the
                    calculating chunks using the change point method
 --source-model     Set if using both 1 and 2 multiples of the frequency and requiring the use of the original source
                    model parameters from Jones, MNRAS, 402 (2010)

 Benchmarking:
 --time-it          Set if wanting to time the various parts of the code. A output file with the "outfile" filename
                    appended with "_timings" will contain the timings
 --sampleprior      (UINT4) Set this to be a number of samples generated from the prior. The nested sampling will not
                    be performed
\end{verbatim}
\end{footnotesize}
\twocolumngrid

\subsection{Examples}

Here we show a few real examples of using the code for a variety of situations. In all cases there data are a minimum of
three files that must be specified to run the code:
\begin{enumerate}
 \item one (or more) data files containing a complex time series. Files must be ascii text format with three
 whitespace-seperated columns
 specifying the data timestamp in GPS seconds, the real part of the data, and the imaginary part of the data, e.g.
 \verb|900000000 1.2564374e-22 -4.5764347e-22|. The file can contain comments if the comment lines are started with
 a \verb|%| or \verb|#|. Gzipped versions of these files can also be input.
 \item a pulsar parameter file. This should be an ascii \sc{tempo2}-stype \verb|.par| file containing information on the
 source's parameters, e.g. sky location and frequency, as generated by \sc{tempo2} \citep{2006MNRAS.369..655H}
 from pulsar pulse time-of-arrival observations. An example \verb|.par| file is given below.
 \item a prior distribution file. This should be an ascii file containing information on the prior distributions
 required for any variable parameters that the code needs to sample posterior probability distribtions for. An example
 prior file is given below.
\end{enumerate}




\subsubsection{Example 1: single detector, single harmonic}

One of the simplest cases for running the code is on data from a single detector and at a single frequncy harmonic (in this
case at twice the rotation frequency).
