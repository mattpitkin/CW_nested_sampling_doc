\section{Conclusions}

In this document we have described some of the workings of a code, \lppenf, that is used for performing parameter estimation and Bayesian model
selection in searches for \gw signals from known pulsars. We have described the available signal models, likelihood functions and prior functions.
We have compared the code to one that had been previously used for the same purpose, \lppef, and found that the results agree very well for a range
of situations. The comparisons have not been completely exhaustive, and there are situations that have not been tested, however, it is worth noting
that \lppe is not necessarily expected to provide complete ground truth. The code has also be validated on additional software and hardware simulated
signals and appears to return posterior distributions that behave as expected, and evidence values that show the expected behaviour.

As a test of the internal nested sampling algorithm, and proposal distributions, used by the code a large number of simulations were performed using
a simple Gaussian likelihood function. These interestingly highlighted that if using, in particular, the ensemble walk proposal distribution the evidence
values output became systematically more biases away from the true value as a function of information gain between the prior and posterior distributions.
The test showed that this bias was alleviated by including a proposal distribution that drew values from the full prior a certain fraction of the time.
The reason for this systematic bias is currently unknown, but in general, given the magnitude of the bias, it should have a relatively minor effect. It was
also shown that with the ensemble walk proposal source amplitude upper limits could be systematically biased by about 2\%, however, again this effect
appears to be alieviated with the flat proposal distribution.

Some example use cases of \lppen are documented in Appendix~\ref{app:usage}.